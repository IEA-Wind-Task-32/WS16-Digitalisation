\documentclass[twocolumn,onesided,9pt]{article}

\PassOptionsToPackage{table}{xcolor}

\usepackage{./Task32FlyerLatexStyle/Task32Flyer}
\usepackage{todonotes}

%% -----------------------------------
%% Document information
%% -----------------------------------
\def\pubdate{29 March 2021}
\title{WS16: Digitalisation of Wind Lidar}
\shorttitle{WS16: Digitalisation}
\doctype{IEA Wind Task 32 Workshop Minutes}
\DOI{10.5281/zenodo.xxxxxx}
\addbibresource{bibliography.bib}

%% -----------------------------------
%% Style modifications (doc specific)
%% -----------------------------------
% task 32 action box
\usepackage{tcolorbox}
\newtcolorbox{taskactions}[1][]
{
	every float=\centering,
	width=1.0\columnwidth,
	boxsep=0pt,
	left=3pt,
	right=3pt,
	top=3pt,
	colframe = Task32Blue2,
	#1,
}
% no indents
\setlength{\parindent}{0pt}
% long tables
\usepackage{supertabular}
% define figure and section references
\newcommand{\fref}[1]{Fig.~\ref{#1}}
\newcommand{\sref}[1]{\S~\ref{#1}}
% set the TOC depth
\setcounter{tocdepth}{2}
% reduce spacing in TOC
\usepackage[titles]{tocloft}
\setlength{\cftbeforesecskip}{3pt}
% authors
\newcommand{\orcid}[1]{\href{https://orcid.org/#1}{\includegraphics[width=12pt]{graphics/ORCIDiD_icon128x128.png}}}
\usepackage{fontawesome}
\newcommand{\mailme}[1]{\href{mailto:#1}{\faicon{envelope-o}}}

%% ===================================
%:
%% Document starts
%% ===================================
\begin{document}

%% -----------------------------------
%% Title
%% -----------------------------------
\maketitle
\thispagestyle{cover}

%% -----------------------------------
%% Authors
%% -----------------------------------
\noindent\begin{minipage}{\columnwidth}
\textcolor{TextLightGrey}{Authors: Andrew Clifton~\orcid{0000-0001-9698-5083}~\mailme{clifton@ifb.uni-stuttgart.de}, %
Ines Würth~\orcid{},
David Schlipf~\orcid{}}
\end{minipage}
\vskip 6pt

%% -----------------------------------
%% Introductory text
%% -----------------------------------
{\Large\noindent%
What will digitalisation mean for wind lidar, and how can digitalised wind lidar be integrated with the digitalised wind energy business? 
}
\vskip 6pt

Wind energy and many other users of wind lidar are going digital, leveraging decades of developments in programming, more reliable communications, and the internet of things. This digital transformation will lead to new ways of working, new opportunities, and new business ideas. But it will also mean that wind lidar will change as well. Together, the wind energy industry and wind lidar will undergo digitalisation.

This workshop used a combination of presentations, group work, and discussions to explore what digitalisation might mean for wind lidar hardware, software, users, and stakeholders.

\tableofcontents

\section*{Disclaimer}
The presence of a person’s name or an organisation's name in this document (e.g., in the list of participants in Table \ref{tab:participants}) should not be taken to imply that that person or organisation agrees with any of the opinions set out here. Similarly, the use of any brand names or trade marks does not constitute an endorsement.

\section*{Attribution}
The workshop ran under modified Chatham House rules. Specifically, participants are free to use information from the discussion, but are not allowed to reveal who made any comment. Material presented by the three invited speakers (listed in “Digitalisation in Practice”) may be attributed.

\section{Agenda}

\begin{table}[!h]
 \centering
 % set up banded rows for the agenda and add lines to the columns
 \arrayrulecolor{Task32Blue2!15}
 \rowcolors{2}{Task32Blue2!5}{white}
 \begin{tabular}{@{}|p{0.125\columnwidth}|p{0.85\columnwidth}|@{}}
 \rowcolor{Task32Blue2} \textbf{Time} & \textbf{Activity} \\
 14:00 & Introduction \\
 14:20 & What is digitalisation?
  \begin{itemize}
        \item How the common lidar data format improved our data processing (Ines Würth, SWE)
        \item The Smart Lidar Concept - New Opportunities for the Lidar Community (David Schlipf)
	\item Modularising wind lidar (Andy Clifton)
    \end{itemize} \\
 15:00 & Working groups: user stories for different wind lidar scenarios\\
 16:00 & Sharing results\\
 16:45 & Close \\
 \end{tabular}
 \label{tab:agenda}
\end{table}


\input{sections/00_introduction}
\input{sections/01_inPractice}

\input{sections/02_userStories}
\input{sections/02_01_wra}
\input{sections/02_02_offshoreForecasting}
\input{sections/02_03_windPlantControl}
\input{sections/02_04_lidarSolutionsProvider}

\section{Summary}

The following section is a summary of the workshop and was prepared by the Operating Agent after the event.

\subsection{Priorities for 2021}

The following were seen as priorities to enable the digitalisation of wind lidar:

\begin{enumerate}
\item
  \textbf{Data standards} were required to enable wind lidar to be used for wind resource assessment, forecasting, wind plant controls, and to enable flexible, modular lidar.
\item
  \textbf{Data flows} to simplify data transfer from lidar devices to other devices and to users. This would be easier with data standards.
\item
  \textbf{A common and modular lidar interface} to enable data input and  output, and control of the wind lidar.
\item
  \textbf{Faster tools} that can be used as part of wind lidar-based   processes, e.g., for energy forecasting.
\item
  \textbf{Energy market flexibility} to allow new business models based on faster reaction times or greater flexibility, e.g., 5- to 10-minute-scale energy forecasting.
\item
  \textbf{Economic models} for different applications that demonstrate
  the economic case for investing in lidar.
\end{enumerate}

Other longer-term needs were identified for each scenario.

\subsection{Potential barriers to digitalisation}

\begin{enumerate}
\item  
  \textbf{Complexity}. Digitalisation is not easy and is a change from
  today's processes.
\item
  \textbf{Standards} that are too specific and incompatible.
\item
  \textbf{Market regulations} and the difficulty of establishing
  reliable, timely data flows.
\item
  \textbf{Data privacy and security issues}, including unwillingness to
  share intellectual property.
\item
  \textbf{Lack of budget for development}, limiting the scope for
  businesses to explore digitalisation.
\item
  \textbf{Lack of competition} to encourage change and new business
  models.
\end{enumerate}

\subsection{What can Task 32 do?}

This workshop showed that there are several things that IEA Wind Task 32 can do to support the digitalisation of wind lidar and it's integration into a digitalised wind energy business.

\begin{enumerate}
\item
  \textbf{Push data standards}. Some nascent data standards already exist,
  for example the e-WindLidar data format \cite{nikola_vasiljevic_2018_2478051} and \href{https://github.com/e-WindLidar/Lidaco}{the Lidaco data converters}. However, these only exist for line-of-sight data   and need to be extended to include processed wind lidar data. 
  \begin{taskactions}
    Task 32 will work with the developers and users of the e-WidLidar data format to extend it.
  \end{taskactions}
\item
  \textbf{Provide examples.} It is not always clear how digitalisation might work. Detailed examples for real use cases will help show the technology and processes that are required, and help understand the costs and benefits of digitalisation.
  \begin{taskactions}
    Task 32 will set up some examples of modular, multi-party collaborative data processing.
  \end{taskactions}
\item  
  \textbf{Encourage collaborative and open R\&D projects}. Task 32 members are
  already heavily involved with low-TRL projects that rely heavily on
  wind lidar data. The results from these projects need to be shared.
  And, where possible, the foundational tools that are developed should
  be shared with the rest of the community to help establish the
  infrastructure and market needed for digitalisation.
  \begin{taskactions}
    Task 32 will continue to provide a platform for the international wind lidar R\&D community to meet and exchange ideas and experience
  \end{taskactions}
\item
  \textbf{Collaborate with other IEA Wind Tasks}. Some Task 32 members
  are also involved with other relevant initiatives, for example IEA
  Wind Task 43 on the digitalisation of wind energy. 
  \begin{taskactions} 
  Task 32 will work with other stakeholder groups to explore how digitalised wind lidar would interface with
  other parts of the wind energy community.
  \end{taskactions}
\end{enumerate}

\section{Conclusions}
The results from this workshop are similar to those seen for studies of digitalisation in other areas of the wind energy industry. They include:

\begin{itemize}
    \item
    Digitalisation happens from the bottom up when users try to automate or reuse old processes, or to share them with colleagues. This can lead to competing, incompatible activities. We may be able to avoid this for wind lidar by leveraging common data formats at different parts of the process, for example the e-windLidar formats \citep{nikola_vasiljevic_2018_2478051}.
    \item 
    Digitalisation can also be top-down, for example by tasking internal teams or by buying in services. This can lead to an adoption problem, that can be avoided by working together with users to create the tools they need, and train them to use them.
    \item
    However it happens, digitalisation needs to be treated as an important (or even strategic) change that can heavily impact users.
    \item
    Like many businesses, the wind lidar business will become modular. Users will increasingly create their own processes based on a mixture of hardware and software tools. 
    \item 
    Service providers - hardware vendors, consultants, researchers - therefore need to work on simplifying the interfaces between their parts of the process.
    \item
    Standards will help with many aspects of digitalisation, as would data and app marketplaces.
    \item
    None of this will happen without management support and encouragement.
    \item 
    We need ways to talk about the costs and benefits of digitalisation.
\end{itemize}

\begin{taskactions}
IEA Wind Task 32 will be convening a working group to make progress on some of these issues. Please get in contact if you would like to take part.
\end{taskactions}

%% -----------------------------------
%% References
%% -----------------------------------
%\section*{References}
% bibliography
\label{sec:References}
\addcontentsline{toc}{section}{References}
{\small
	\printbibliography
}
\vspace*{\fill}

%% -----------------------------------
%% Outlined block of smaller text
%% -----------------------------------
\begin{tcolorbox}[width=1.0\columnwidth,
		boxsep=0pt,
		left=3pt,
		right=3pt,
		top=3pt,
		arc=0pt,
		boxrule=0.5pt,
		toprule=0.5pt,
		colback=white,
		coltext=TextGrey
	]
	{\footnotesize

		%% -----------------------------------
		%% IEA WIND AND TASK 32
		%% -----------------------------------
		
		\begin{tabular}{m{0.3\columnwidth}m{0.6\columnwidth}}
			\rowcolor{white}
			\multicolumn{2}{p{0.9\columnwidth}}{%
			This document was self published by IEA Wind Task 32.
			}\\
			% IEA Wind * DO NOT EDIT THIS TEXT *
			\rowcolor{white}
			\includegraphics[height=2cm]{graphics/IEAWind_logo.jpg}  &
			The International Energy Agency is an autonomous organisation which works to ensure reliable, affordable and clean energy for its 30 member countries and beyond. The IEA Wind Technology Collaboration Programme supports the work of 38 independent, international groups of experts that enable governments and industries from around the world to lead programmes and projects on a wide range of energy technologies and related issues.%
			\\
			% Task 32 * DO NOT EDIT THIS TEXT *
			\rowcolor{white}
			\includegraphics[height=1.5cm]{graphics/Task32_logo.jpg} &
			\href{https://community.ieawind.org/task32/home}{IEA Wind Task 32} exists to identify and mitigate the barriers to the deployment of wind lidar for wind energy applications.\\
			% authors, etc
			\rowcolor{white}
			\multicolumn{2}{p{0.9\columnwidth+2\tabcolsep}}{%
		%% -----------------------------------
		%% For more information
		%% -----------------------------------
		% N.B. do not add line breaks between the next items
		\textbf{For more information:} See the  \href{https://community.ieawind.org/task32/home}{Task 32 website}.
		%% -----------------------------------
		%% Authors
		%% -----------------------------------
		\textbf{Author team:} %
		Andrew Clifton (Task 32 Operating Agent, University of Stuttgart, Germany), %
		Ines Würth (SWE, University of Stuttgart, Germany), %		
		David Schlipf (Task 32 operating Agent, Flensburg University of Applied Sciences, Germany).
		%% -----------------------------------
		%% Reviewers
		%% -----------------------------------
		%\textbf{Reviewers:} %
		% first last (short affiliation), %
		% first last (short affiliation).
		%% -----------------------------------
		%% Images
		%% -----------------------------------
		\textbf{Images:}
		Banner, left to right: \href{https://unsplash.com/@alexkixa}{Alexandre Debiève on Unsplash}, \href{http://ifb.uni-stuttgart.de}{SWE U. Stuttgart}, \href{https://unsplash.com/@markusspiske}{Markus Spiske on Unsplash}.
	}\\
	\end{tabular}%

	}

	%% -----------------------------------
	%% End of highlighted block
	%% -----------------------------------
\end{tcolorbox}
\vspace*{\fill}

\clearpage
\begin{table*}[!h]
 \centering
 \caption{Actors in today's lidar-based wind resource assessments}
 % set up banded rows for the agenda and add lines to the columns
 \arrayrulecolor{Task32Blue2!15}
 \rowcolors{2}{Task32Blue2!5}{white}
 \begin{tabular}{@{}|p{0.125\textwidth}|p{0.185\textwidth}|p{0.185\textwidth}|p{0.185\textwidth}|p{0.185\textwidth}|@{}}
 \rowcolor{Task32Blue2} & \textbf{Alice} & \textbf{Bob} & \textbf{Darla} & \textbf{Cory} \\
They are: & 
    Lidar provider & 
    Data analyst & 
    Data manager & 
    Business / Project Manager \\
Their biggest problem is: & 
    Selling the lidar. 

    Deploying it on time &
    Configuring the lidar

    Uncertainty about metadata &
    Getting data into the organization and managing it. & 
    Knowing when to invest in using lidar. \\
Success for them is: & 
    Successful campaign with feedback from other campaigns &
    Good verification \& validation & 
    No issues during LiDAR data measurement &
    Clearer guideline on when investing in Lidar makes sense;

    Financing as planned \\
Problem statement & 
    \multicolumn{4}{p{0.74\textwidth+6\tabcolsep+3\arrayrulewidth}}{Need to get a ``good enough'' measurement for funding (enough data with sufficient uncertainty)} \\
\end{tabular}
\label{tab:01_wra_now}
\end{table*}
% note that the length of the multicolumn is managed by https://tex.stackexchange.com/a/204917

\begin{table*}[!h]
 \centering
 \caption{Actors in near-future lidar-based wind resource assessments}
 % set up banded rows for the agenda and add lines to the columns
 \arrayrulecolor{Task32Blue2!15}
 \rowcolors{2}{Task32Blue2!5}{white}
 \begin{tabular}{@{}|p{0.125\textwidth}|p{0.185\textwidth}|p{0.185\textwidth}|p{0.185\textwidth}|p{0.185\textwidth}|@{}}
 \rowcolor{Task32Blue2} & \textbf{Alice} & \textbf{Bob} & \textbf{Darla} & \textbf{Cory} \\
They are: & 
    Lidar provider & 
    Data analyst & 
    Data manager & 
    Business / Project Manager \\
Their biggest problem is: & 
    Selling the lidar. 
    
    Deploying it on time &
    Configuring the lidar; uncertainty about metadata & 
    Getting data into the organization and managing it. & 
    Knowing when to invest in using lidar \\
Success for them is: & 
    Successful campaign with feedback from other campaigns & 
    Good verification \& validation & 
    No issues during LiDAR data measurement & 
    Financing as planned / Clearer guideline on when investing in Lidar makes sense \\
Digitalisation helps by: & 
    Providing info & 
    Validation reports visual & 
    Smoother installation \& data management &
    Enabling Live queries of the project data and results \\
What problems are left? & 
    Maintaining standards & 
    Visual communication / reports & 
    Monitoring the lidar function, data storage, and security & 
    Uncertainty;

    ROI not clear 
\end{tabular}
\label{tab:01_wra_future}
\end{table*}
\clearpage
\begin{table*}[!h]
 \centering
 \caption{Actors in energy forecasting for offshore wind farms today}
 % set up banded rows for the agenda and add lines to the columns
 \arrayrulecolor{Task32Blue2!15}
 \rowcolors{2}{Task32Blue2!5}{white}
 \begin{tabular}{@{}|p{0.125\textwidth}|p{0.255\textwidth}|p{0.255\textwidth}|p{0.255\textwidth}|@{}}
 \rowcolor{Task32Blue2} & \textbf{Alice} & \textbf{Bob} & \textbf{Chris}  \\
They are: & 
    Wind farm operator & 
    Forecaster & 
    Energy trader \\
Their biggest problem is: & 
    Unknown wind conditions in the future & 
    No data & 
    If they cannot deliver sold energy \\
Success for them is: & 
    Wants to sell her energy & 
    Having low uncertainty forecast & 
    Highest revenue \\
Problem statement: & 
    \multicolumn{3}{|p{0.765\textwidth+4\tabcolsep+2\arrayrulewidth}|}{We need the best wind data to produce a low-uncertainty energy yield forecast}\\
\end{tabular}
\label{tab:02_offshoreforecasting_now}
\end{table*}


\begin{figure*}
    \centering
    \fbox{
    \includegraphics[width=0.85\textwidth]{figures/02_offshoreforecasting.png}
    }
    \caption{A vision for future energy trading for offshore wind}
    \label{fig:02_offshoreforecasting_future}
\end{figure*}

\clearpage
\input{sections/02_03_windPlantControl_figures}
\clearpage

\begin{table*}[!h]
    \centering
    \caption{Actors in providing lidars today}
    % set up banded rows for the agenda and add lines to the columns
    \arrayrulecolor{Task32Blue2!15}
    \rowcolors{2}{Task32Blue2!5}{white}
    \begin{tabular}{@{}|p{0.125\textwidth}|p{0.255\textwidth}|p{0.255\textwidth}|p{0.255\textwidth}|@{}}
    \rowcolor{Task32Blue2} & Alice & & Chris \\
    They are: &
        Provider &
        &
        Customer \\
    Their biggest problem is: &
        Provide the right device to Lisa & 
        &
        Can't correctly interpret the data or get out needed data \\
    Success for them is: &
        Provide the right device for the right price &
        &
    Long term use of the device \\
    Problem statement: &
        \multicolumn{3}{|p{0.765\textwidth+4\tabcolsep+2\arrayrulewidth}|}{The connection between the provider and customer is missing}
\end{tabular}
\label{tab:04_lidarSolutionsProvider_now}
\end{table*}

\begin{figure*}
    \centering
    \fbox{
    \includegraphics[width=0.85\textwidth]{figures/04_lidarSolutionsProvider.png}
    }
    \caption{Actors in a future lidar market}
    \label{fig:02_offshoreforecasting_future}
\end{figure*}


\begin{table*}[!h]
    \centering
    \caption{Actors in providing lidars in the future}
    % set up banded rows for the agenda and add lines to the columns
    \arrayrulecolor{Task32Blue2!15}
    \rowcolors{2}{Task32Blue2!5}{white}
    \begin{tabular}{@{}|p{0.125\textwidth}|p{0.255\textwidth}|p{0.255\textwidth}|p{0.255\textwidth}|@{}}
    \rowcolor{Task32Blue2} & Alice & Bob & Chris \\
    They are... & 
        Lidar Provider & Data Provider & Customer \\
    Their biggest problem is: &
        Selling less lidars because they will be rented &
        Handling different lidar data & 
        Need to define clearly what type of data is needed \\
    Success for them is: & 
        Operating a fleet of lidars for customer & 
        Get a common method to analyze data from different lidars & 
        More time for data analysis \\
    Digitalisation helps by: &
        Managing and monitoring lidars, gaining knowledge &
        Compare data

        Finding the most suitable lidar depending on site characteristics/ino want to get out &
        Providing insights and solutions faster. No need for handling lidar by additional staff. \\
    Any problems left? & 
        Have to stand out among other lidar providers.

        Modularity of software and hardware & 
        Defining tasks of data provider, common data and interface formats. 
    
        Managing lots of lidars effectively. &
        Who owns the data (only licence for usage or ownership)?
\end{tabular}
\label{tab:04_lidarSolutionsProvider_future}
\end{table*}

\clearpage
\input{sections/05_participants}

\end{document}
