\section{User stories for wind lidar digitalisation}

The workshop participants next split into small groups to create user stories for four different situations. These situations were proposed by the workshop organiser as representative of common situations:
\begin{itemize}
\item Wind resource assessment
\item Monitoring offshore wind turbines
\item Lidar for wind plant control
\item Lidar solution providers
\end{itemize}

The groups further defined the scenario and the current actors. They identified how these actors interact and what challenges they might face. They then considered how that scenario might change over the next three to five years, and how digitalisation might impact those scenarios. Finally, each group identified any potential “show-stoppers” and the highest priority activities to enable success. The results are summarised in the following subsections and are based on a template provided by the Workshop organiser. The results have not been edited and may include spelling errors or be unclear.

\begin{taskactions}
    Please report any corrections to the \href{mailto:ieawind.task32@ifb.uni-stuttgart.de}{Task 32 Operating Agent}.
\end{taskactions}