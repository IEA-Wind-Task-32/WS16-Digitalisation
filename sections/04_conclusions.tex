\section{Conclusions}
The results from this workshop are similar to those seen for studies of digitalisation in other areas of the wind energy industry. They include:

\begin{itemize}
    \item
    Digitalisation happens from the bottom up when users try to automate or reuse old processes, or to share them with colleagues. This can lead to competing, incompatible activities. We may be able to avoid this for wind lidar by leveraging common data formats at different parts of the process, for example the e-windLidar formats \citep{nikola_vasiljevic_2018_2478051}.
    \item 
    Digitalisation can also be top-down, for example by tasking internal teams or by buying in services. This can lead to an adoption problem, that can be avoided by working together with users to create the tools they need, and train them to use them.
    \item
    However it happens, digitalisation needs to be treated as an important (or even strategic) change that can heavily impact users.
    \item
    Like many businesses, the wind lidar business will become modular. Users will increasingly create their own processes based on a mixture of hardware and software tools. 
    \item 
    Service providers - hardware vendors, consultants, researchers - therefore need to work on simplifying the interfaces between their parts of the process.
    \item
    Standards will help with many aspects of digitalisation, as would data and app marketplaces.
    \item
    None of this will happen without management support and encouragement.
    \item 
    We need ways to talk about the costs and benefits of digitalisation.
\end{itemize}

\begin{taskactions}
IEA Wind Task 32 will be convening a working group to make progress on some of these issues. Please get in contact if you would like to take part.
\end{taskactions}