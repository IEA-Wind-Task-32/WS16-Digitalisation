\subsection{Wind resource assessment}

This group considered a scenario in which a wind plant developer wants to deploy a wind lidar for a wind resource assessment campaign. The users include the lidar provider and several customers, including a data analyst, a data manager, and a project manager (Table \ref{tab:01_wra_now}).

This group felt that the biggest potential from digitalisation was the ability to feed back experience from measurement campaigns into the next campaign, and thus build up repeatable, high-quality wind resource assessment processes.

The group expected feedback to be enabled in the near future by the availability of more data and better data storage and retrieval. They expected the same actors still to be present in a wind resource assessment campaign, and with similar concerns as to today (Table \ref{tab:01_wra_future}).

The group identified the following things as being important to making this vision a reality:

\begin{enumerate}
\item
  \textbf{A data hub}: software and services to receive, manage and
  process data. This should be based on community standards and should
  be delivered by the lidar provider(s) and the community. It was noted
  that this should not be too complex, otherwise it might prevent
  adoption.
\item
  \textbf{Data standards}: clear standards for data and data products at
  different stages of the wind resource assessment process are required.
  They should be compatible with all sections of the wind energy
  industry. This should be delivered by the wind lidar and wind energy
  community. These were also seen as a priority.
\end{enumerate}
