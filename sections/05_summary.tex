\section{Summary}

The following section is a summary of the workshop and was prepared by the Operating Agent after the event.

\subsection{Priorities for 2021}

The following were seen as priorities to enable the digitalisation of wind lidar:

\begin{enumerate}
\item
  \textbf{Data standards} were required to enable wind lidar to be used for wind resource assessment, forecasting, wind plant controls, and to enable flexible, modular lidar.
\item
  \textbf{Data flows} to simplify data transfer from lidar devices to other devices and to users. This would be easier with data standards.
\item
  \textbf{A common and modular lidar interface} to enable data input and  output, and control of the wind lidar.
\item
  \textbf{Faster tools} that can be used as part of wind lidar-based   processes, e.g., for energy forecasting.
\item
  \textbf{Energy market flexibility} to allow new business models based on faster reaction times or greater flexibility, e.g., 5- to 10-minute-scale energy forecasting.
\item
  \textbf{Economic models} for different applications that demonstrate
  the economic case for investing in lidar.
\end{enumerate}

Other longer-term needs were identified for each scenario.

\subsection{Potential barriers to digitalisation}

\begin{enumerate}
\item  
  \textbf{Complexity}. Digitalisation is not easy and is a change from
  today's processes.
\item
  \textbf{Standards} that are too specific and incompatible.
\item
  \textbf{Market regulations} and the difficulty of establishing
  reliable, timely data flows.
\item
  \textbf{Data privacy and security issues}, including unwillingness to
  share intellectual property.
\item
  \textbf{Lack of budget for development}, limiting the scope for
  businesses to explore digitalisation.
\item
  \textbf{Lack of competition} to encourage change and new business
  models.
\end{enumerate}

\subsection{What can Task 32 do?}

This workshop showed that there are several things that IEA Wind Task 32 can do to support the digitalisation of wind lidar and it's integration into a digitalised wind energy business.

\begin{enumerate}
\item
  \textbf{Push data standards}. Some nascent data standards already exist,
  for example the e-WindLidar data format \cite{nikola_vasiljevic_2018_2478051} and \href{https://github.com/e-WindLidar/Lidaco}{the Lidaco data converters}. However, these only exist for line-of-sight data   and need to be extended to include processed wind lidar data. 
  \begin{taskactions}
    Task 32 will work with the developers and users of the e-WidLidar data format to extend it.
  \end{taskactions}
\item
  \textbf{Provide examples.} It is not always clear how digitalisation might work. Detailed examples for real use cases will help show the technology and processes that are required, and help understand the costs and benefits of digitalisation.
  \begin{taskactions}
    Task 32 will set up some examples of modular, multi-party collaborative data processing.
  \end{taskactions}
\item  
  \textbf{Encourage collaborative and open R\&D projects}. Task 32 members are
  already heavily involved with low-TRL projects that rely heavily on
  wind lidar data. The results from these projects need to be shared.
  And, where possible, the foundational tools that are developed should
  be shared with the rest of the community to help establish the
  infrastructure and market needed for digitalisation.
  \begin{taskactions}
    Task 32 will continue to provide a platform for the international wind lidar R\&D community to meet and exchange ideas and experience
  \end{taskactions}
\item
  \textbf{Collaborate with other IEA Wind Tasks}. Some Task 32 members
  are also involved with other relevant initiatives, for example IEA
  Wind Task 43 on the digitalisation of wind energy. 
  \begin{taskactions} 
  Task 32 will work with other stakeholder groups to explore how digitalised wind lidar would interface with
  other parts of the wind energy community.
  \end{taskactions}
\end{enumerate}

These activities will be included in future versions of the Task 32 Roadmap \cite{clifton_2020_roadmap}.
