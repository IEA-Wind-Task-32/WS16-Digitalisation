\subsection{Panel discussion: \enquote{Wind lidar - I wish we knew how to...}}

We started with presentations from all panelists with their views
of where the entire wind energy and wind lidar community have work to
do. 52 people joined us.

\subsubsection{Presentations}

Mads V. Sorensen: I wish I knew how to get most value out of short (e.g. 3 months) measurement campaigns

\begin{itemize}
    \item in terms of TI, seasonality, shear
    \item why should one use a lidar if it is the same cost for 12 months than a mast
    \item why not use the full advantage of the lidar
\end{itemize}

Peter Rosenbusch: I wish WE knew how to\ldots{}

\begin{itemize}
    \item Eliminate the cup anemometer from the uncertainty budget
    \item Augment acceptance of ground-based lidars in complex terrain
    \item Establish nacelle lidar for PPT in complex terrain
    \item Optimize offshore WRA by combining floating lidar and lidar from the shore
    \item Establish best practice for site suitability with ground based lidars
\end{itemize}

Zachary Parker: I wish we knew how to\ldots{}

\begin{itemize}
    \item determine load assessment bias and uncertainty given remote sensing measurements
    \item validate, correct and use remote sensing data for load assessment → TI, shear and wind speed
    \item provide guidance from the turbine OEM perspective on the use (or not) of remote sensing
\end{itemize}

Julia Gottschall: I wish we knew how to\ldots{}

\begin{itemize}
    \item Do the optimal measurements (most likely with lidar)... in terms of chosen technology, setup, duration, requirements on accuracy and availability → what should we really measure?
    \item There are two necessary steps: 1, to understand application as well as possible and 2. to consider all possible data sources
\end{itemize}

\subsubsection{Discussions}

\emph{Many of the following questions and chat were taken verbatim from the video chat window. There have been some edits for spelling and clarity.}

Question from Julia: Should we put a scanning lidar on a buoy?

\begin{itemize}
\item  Peter Rosenbusch: I have no objections to this. We are involved in a
  research project. The definition of a scanning lidar is a device which
  can point the measurement to any point. A benefit of a scanning lidar
  is to be able to put it on the shore, or on the transition piece of a
  turbine.
\end{itemize}

From an industry researcher to Julia: Should we 'measure' turbulence using TI as currently defined involving the standard deviation over 600-second intervals, or is there some other way to 'measure' turbulence that would give a better input to models? Would TKE be better to use in conjunction with models?

\begin{itemize}
\item  Julia asks back: is it easier to work on the measurements or on the models? I personally don't know. We should not force a lidar to work as a cup because it cannot. We should understand the models and the measurements better. We also need to consider the bankability and the industry. My conclusion is we should try all of this, even a small impact will have a larger impact in the future. We should not be happy with using a lidar with a standard TI.
\item  Researcher: the measurement people stay with what they know and same for the load assessment people. Both groups should work together better. I think the load assessment process will be very difficult to change as it is based on many years of understanding of how to calibrate the load models. The new way of lidar measurement would require to throw away the existing experience. In the short term, you should adapt the measurements. In the long term, you should adapt the process.
\item  Peter: the calibration of the models to a point measurement seems less perfect in light of always growing turbines. 
\item Zachary: we see if we just use the lidar as a point measurement, we just get higher loads. We really need to understand first how to use the additional  information.
\item  David Schlipf: a lidar can give you a much better estimate over the whole rotor area than a cup anemometer could.
\end{itemize}

From an industry engineer to Mads and the group: Do we have a method to 'long-term' correct standard deviation / TI...from 3 months to 1 or multiple years? How to get the most out of your measurements?

\begin{itemize}
\item  Answer: Not really!
\item  Andy: this ties in to the presentations yesterday, especially from Reesa. We need to develop new tools, but the need for simple tools is very clear. We cannot treat this just as an academic problem, we need simple solutions.
\end{itemize}

Andy wants to come back to question of whether we get the most value out of a lidar, or could we do better?

\begin{itemize}
\item  Peter: I think we could do better. We are trying to optimize e.g. the position of the lidar. Do you have simulation tools to help you decide whether to put that?
\item  Mads: there are flow models that can help, but they come at a cost. If you're not sure you only start measuring at one point. I would like to take the idea of the modeling: to get the most out of lidar measurements, the effort should be on the modeling. E.g. you could throw information from several different measurements positions into a flow model and get better results.
\item  Zachary: the lidar can give you information on the stability, and this   is very important to get the modeling right.
\item  Julia: There are a lot of statistics of the wind fields involved, it is not just the modeling that is a challenge. So we should invest a lot of work in both. German guidelines will stick to 12 months for site assessment. I think it depends on the site. 
\item  Zachary: There are a lot of statistics coming out of the lidar, we should also look more at the raw data.
\item  Andy: We have made some progress in the last few years on measurements and modeling, but there is still a lot of work to do.
\end{itemize}

Folks who leave the meeting should do this..

\begin{itemize}
\item  Mads: consider the measurement period
\item  Julia: understand what your colleagues want to use the data for
\item  Zachary: study colocated lidar and sonic data with 1Hz
\item  Peter: brainstorm how to use the flexibility that a lidar provides
\end{itemize}